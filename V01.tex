\chapter{Geraden und Ebenen}
Der n-Dimensionale Raum
$R^n$ \{x =  \left( \begin{array}[c] x_1 \\ x_2 \\ \vdots \\ x_n \end{array} \right) \}
Elemente von $R^n$ werden als Vektoren bezeichnet.
$R^2$ % Figure Koordinaten System 
$x = \left( \begin{array}{c}2 \\ 1\end{array} \right)$\\
$y = \left( \begin{array}{c}3 \\ -2\end{array} \right)$\\
$x + y = \left( \begin{array}{c}5 \\ -1\end{array} \right)$
\section*{Vektoradition}
$\left( \begin{array}{c}x_1 \\ \vdots \\ x_n\end{array} \right)$ + 
$\left( \begin{array}{c}y_1 \\ \vdots \\ y_n\end{array} \right)$ = 
$\left( \begin{array}{c}x_1 + x_2 \\ \vdots \\ x_n + y_n\end{array} \right)$
Multiplikation eines Vektors mit einer Zahl: "Skalieren"
$\lambda \in \R$\\
$\lambda \cdot \left( \begin{array}{c}x_1 \\ \vdots \\ x_n\end{array} \right) = \left( \begin{array}{c}\lambda \cdot x_1 \\ \vdots \\ \lambda \cdot x_n\end{array} \right)$\\
% Figure Koordinaten System
Geraden in $R^2$\\
Durch 2 verschiedene Punkte läuft eine eindeutig bestimmte Gerade
% Figure Koordianten System
Diferenz $x - y = \left( \begin{array}{c}2 \\ 1\end{array} \right) - \left( \begin{array}{c}1 \\ -1\end{array} \right) = \left( \begin{array}{c}1 \\ 2\end{array} \right)$
"Vektor von y u x"\\
$Gerade = g = \{y + \lambda \cdot (y - x) \mid \lambda \in \R\}$\\
Parametrisierung der Geraden\\
Betrachte lineare Gleichung\\
$3x_1 - 2x_2 = 1$\\
$\{x = \left( \begin{array}{c}x_1 \\ x_2\end{array} \right) \in \R \mid 3x_1 - 2x_2 = 1 \} = h$\\
$3x_1 = 2x_2 + 1 \mid x_1 = \frac{1}{3} + \frac{2}{3}x_2$\\
Lösung der Gleichung:\\
$x_2$ frei wählbar, $x_1$ dann festgelegt.\\
$x_2 = \lambda \in \R$\\
$x_1 = \frac{1}{3} \cdot \lambda$\\
$x = \left( \begin{array}{c}\frac{1}{2} + \frac{2}{3} \cdot \lambda \\ \lambda \cdot 1\end{array} \right) = \left( \begin{array}{c}\frac{1}{3} \\ 0\end{array} \right) + \lambda \cdot \left( \begin{array}{c}\frac{2}{3} \\ 1\end{array} \right)$\\
$h = \{\left( \begin{array}{c}\frac{1}{3} \\ 0\end{array} \right) + \lambda \cdot \left( \begin{array}{c}\frac{2}{3} \\ 1\end{array} \right) \mid \lambda \in \R \}$ Parametrisierung einer Geraden\\
Gegeben eine Gerade in Parameterdarstellung\\
$\{ x + \lambda \cdot y \mid \lambda \in \R \} \qquad x, y \in R^2$ fest gewählt.\\
Finde dann eine lineare Gleichung\\
(*) $a_1 z_1 + a_2 z_2 = b \qquad a_1, a_2, b \in \R$\\
alle Vektoren $x + \lambda y$ sollen Lösungen von (*) sein.\\


In $R^3$\\
\def Eine Ebene in $R^3$ ist die Lösungsmenge einer linearen Gleichung $a_1 \cdot x_1 + a_2 \cdot x_2 + a_3 \cdot x_3 = b$ \\
$a_1, a_2, a_3, b \in \R fest, nicht a_1 = a_2 = a_3 = 0$\\
\bsp \\
$E = \left\lbrace \left( \begin{array}{c}x_1\\x_2\\x_3 \end{array} \right) \in \R \mid 2x_1 + 3x_2 + 4x_3 = 6 \right\rbrace$\\
Kann die gleichung nach jeder Variablen auflösen.\\
z.B. $x_1 = 3 - \frac{3}{2}x_2 - 2x_3$\\
$x_2, x_3$ frei wählbar, $x_1$ dann festgelegt.\\
Parameter: $\lambda_1, \lambda_2 \in \R$\\
$x_2 = \lambda_1$\\
$x_3 = \lambda_2$\\
$x_1 = 3 - \frac{3}{2}\lambda_1 - 2 \lambda_2$\\
$x = \left( \begin{array}{c}x_1\\x_2\\x_3 \end{array} \right) = \left( \begin{matrix}
3 & -\frac{3}{2}\lambda_1 & -2 \lambda_2 \\
  &  \lambda_1 & \\
  &  & \lambda_2\\
\end{matrix} \right) = \left( \begin{array}{c}3\\0\\0 \end{array} \right) + \lambda_1 \left( \begin{array}{c}-\frac{3}{2}\\1\\0 \end{array} \right) + \lambda_2 \left( \begin{array}{c}-2\\0\\1 \end{array} \right)$\\
Bessere Zeichnung: Schneide Ebenen mit den Achsen\\
$E \cap (x_1-Achse) = \{ \left( \begin{array}{c}x_1\\0\\0 \end{array} \right) \mid 2x_1 = 6 \} = \{ \left( \begin{array}{c}3\\0\\0 \end{array} \right) \}$
$E \cap (x_2-Achse) = \{ \left( \begin{array}{c}0\\2\\0 \end{array} \right) \}$\\
$E \cap (x_3-Achse) = \{ \left( \begin{array}{c}0\\0\\\frac{2}{3} \end{array} \right) \}$\\
Schnitt von zwei Ebenen